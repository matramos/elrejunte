\newpage
\section{Juegos}
\subsection{Nim Game}
\begin{flushleft}
Juego en el que hay N pilas, con objetos. Cada jugador debe sacar al menos un objeto de una pila. GANA el jugador que saca el \'ultimo objeto.
\end{flushleft}
$P_0{\oplus}P_1{\oplus}...{\oplus}P_n = R$
\begin{flushleft}
Si $R{\neq}0$ gana el jugador 1.
\end{flushleft}

\subsubsection{Misere Game}
\begin{flushleft}
Es un juego con las mismas reglas que Nim, pero PIERDE el que saca el \'ultimo objeto. Entonces teniendo el resultado de la suma $R$, y 
si todas las pilas tienen 1 solo objeto $todos1{=}true$, podemos decir que el jugador2 GANA si:
\end{flushleft}
${(R{=}0){\&}{\neg}{todos1}{\parallel}(R{\neq}0 ){\&}{todos1}}$
\subsection{Ajedrez}
\subsubsection{Non-Attacking N Queen}
\begin{footnotesize}
	\textbf{Utiliza:} \texttt{<algorithm>}\\
	\textbf{Notas:} todo es $ O(!N \cdot N^{2})$.
\end{footnotesize}
\lstinputlisting[language=C++]{secciones/juegos/ajedrez_reinas.cpp}
\subsection{Green Hackenbush}
\lstinputlisting[language=C++]{secciones/juegos/green_hackenbush.cpp}
