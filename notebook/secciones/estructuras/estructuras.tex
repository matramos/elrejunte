\section{Estructuras de datos}

\subsection{Disjoint Sets}
\subsubsection{Union Find (OOP)}
\begin{footnotesize}
	\textbf{Utiliza:} \texttt{<vector>}\\
	\textbf{Notas:} Rangos \texttt{[i,j]} (\textit{0 based}). No recomendable si se tienen que crear y destruir muchos objetos. Probar funcionamiento en casos l\'imites. 
\end{footnotesize}
\lstinputlisting[language=C++]{secciones/estructuras/union_find_oop.cpp}
\subsubsection{Union Find (C Style/Static)}
\begin{footnotesize}
	\textbf{Utiliza:} \texttt{<cstring>}\\
	\textbf{Notas:} Rangos \texttt{[i,j]} (\textit{0 based}). En \texttt{init(n)}, \texttt{n $\leq$ MAXN}
\end{footnotesize}
\lstinputlisting[language=C++]{secciones/estructuras/union_find_static.cpp}

