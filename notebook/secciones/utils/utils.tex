\newpage
\section{Utils}
\subsection{Convertir string a num e viceversa}
\lstinputlisting[language=C++]{secciones/utils/string_to_num.cpp}
\subsection{Truquitos para entradas/salidas}
\lstinputlisting[language=C++]{secciones/utils/tricks.cpp}
\subsection{Mejorar Lectura de Enteros}
\lstinputlisting[language=C++]{secciones/utils/mejora_lectura_enteros.cpp}
\subsection{Comparaci\'on de Double}
\lstinputlisting[language=C++]{secciones/utils/comparar_double.cpp}
\subsection{Limites}
\lstinputlisting[language=C++]{secciones/utils/limites.cpp}
\subsection{Iterar subconjuntos}
\lstinputlisting[language=C++]{secciones/utils/iterar_subconjuntos.cpp}
\subsection{Operaciones de bits}
\lstinputlisting[language=C++]{secciones/utils/bits_operations.cpp}
\subsection{Comparator for set}
\lstinputlisting[language=C++]{secciones/utils/comparator_for_set.cpp}
\subsection{Tablita de relacion de Complejidades}
\begin{center}
\tablefirsthead{\hline \textbf{n} & \textbf{Peor AC Complejidad} & \textbf{Comentario} \\ \hline}
\tablehead{\hline \multicolumn{3}{|c|}{\tiny Continuaci\'on}\\ \hline \textbf{n} & \textbf{Peor AC Complejidad} & \textbf{Comentario} \\}
% \tabletail{\hline}
% \tablelasttail{\hline}
\footnotesize
{
  \begin{xtabular}{|C{.15\columnwidth}|C{.25\columnwidth}|L{.55\columnwidth}|}
  \texttt{$\leq[10..11]$}  & \texttt{$O(n!),O(n^6)$}         & ej. Enumerar permutaciones                                                         \\ \hline
  \texttt{$\leq[15..18]$}  & \texttt{$O(2^n\times n^2)$}     & ej. DP TSP                                                                         \\ \hline
  \texttt{$\leq[18..22]$}  & \texttt{$O(2^n\times n)$}       & ej. DP con mascara de bits                                                         \\ \hline
  \texttt{$\leq100$}       & \texttt{$O(n^4)$}               & ej. DP con 3 dimensiones $+ O(n)$ loops                                            \\ \hline
  \texttt{$\leq400$}       & \texttt{$O(n^3)$}               & ej. Floyd Warshall                                                                 \\ \hline
  \texttt{$\leq2K$}        & \texttt{$n^2\log_2n$}           & ej. 2 loops anidados + una busqueda en arbol en una estructura de datos            \\ \hline
  \texttt{$\leq10K$}       & \texttt{$O(n^2)$}               & ej. Ordenamiento Burbuja/Selecci\'on/Inserci\'on                                   \\ \hline
  \texttt{$\leq1M$}        & \texttt{$O(n\log_2n)$}          & ej. Merge Sort, armar Segment Tree                                                 \\ \hline
  \texttt{$\leq100M$}      & \texttt{$O(n),O(\log_2n),O(1)$} & La mayor\'ia de los problemas de contest tiene $n\leq1M$ (cuello de botella en I/O)\\ \hline
  \end{xtabular}
}
\end{center}

  
\subsection{Compilar C++11 con g++}
Dos opciones, \'util en Linux.
\begin{code}
g++ -std=c++11 {file} -o {filename}

g++ -std=c++0x {file} -o {filename}
\end{code}
\subsection{Build de C++11 para Sublime Text}
\input{secciones/utils/c++_sublime.tex}
\subsection{Funciones Utiles}
\begin{tabular}{|l|l|p{4.5cm}|} \hline
\textbf{Algo} & \textbf{Params} &  \textbf{Funcion} \\  \hline
fill, fill\_n & f, l / n, elem & \textit{void} llena [f, l) o [f, \\ && f+n) con elem \\  \hline
lower\_bound, upper\_bound & f, l, elem & \textit{it} al primer / ultimo donde se \\ && puede insertar elem para que\\ && quede ordenada \\  \hline
copy & f, l, resul & hace resul+$i$=f+$i$ $\forall i$ \\  \hline
find, find\_if, find\_first\_of & f, l, elem & \textit{it} encuentra i $\in$[f,l) tq. i$=$elem, \\ & / pred / f2, l2 & pred(i), i$\in$[f2,l2)\\\hline
count, count\_if & f, l, elem/pred & cuenta elem, pred(i)\\\hline
search & f, l, f2, l2 & busca [f2,l2) $\in$ [f,l)\\\hline
replace, replace\_if & f, l, old & cambia old / pred(i) por new \\ & / pred, new &\\\hline
partition, stable\_partition & f, l, pred & pred(i) ad, !pred(i) atras\\\hline
lexicographical\_compare & f1,l1,f2,l2 & \textit{bool} con [f1,l1]<[f2,l2]\\\hline
accumulate & f,l,i,[op] & \textit{T} $=$ $\sum$/oper de [f,l)\\\hline
inner\_product & f1, l1, f2, i & \textit{T} $=$ i $+$ [f1, l1) . [f2, $\ldots$ )\\\hline
partial\_sum & f, l, r, [op] & r+i = $\sum$/oper de [f,f+i] $\forall i \in$[f,l)\\\hline
\_\_builtin\_ffs& unsigned int & Pos. del primer 1 desde la derecha\\\hline
\_\_builtin\_clz & unsigned int & Cant. de ceros desde la izquierda.\\\hline
\_\_builtin\_ctz & unsigned int & Cant. de ceros desde la derecha.\\\hline
\_\_builtin\_popcount & unsigned int & Cant. de 1’s en x.\\\hline
\_\_builtin\_parity & unsigned int & 1 si x es par, 0 si es impar.\\\hline
\_\_builtin\_XXXXXXll & unsigned ll & = pero para long long's.\\\hline
\end{tabular}
